% This is the Reed College LaTeX thesis template. Most of the work
% for the document class was done by Sam Noble (SN), as well as this
% template. Later comments etc. by Ben Salzberg (BTS). Additional
% restructuring and APA support by Jess Youngberg (JY).
% Your comments and suggestions are more than welcome; please email
% them to cus@reed.edu
%
% See http://web.reed.edu/cis/help/latex.html for help. There are a 
% great bunch of help pages there, with notes on
% getting started, bibtex, etc. Go there and read it if you're not
% already familiar with LaTeX.
%
% Any line that starts with a percent symbol is a comment. 
% They won't show up in the document, and are useful for notes 
% to yourself and explaining commands. 
% Commenting also removes a line from the document; 
% very handy for troubleshooting problems. -BTS

% As far as I know, this follows the requirements laid out in 
% the 2002-2003 Senior Handbook. Ask a librarian to check the 
% document before binding. -SN

%% Jay Kruer's includes
\newcommand{\term}{\emph} % TODO: make a tool to grab all the emph's from my
                          % document and optionally add them to the glossary
\newcommand{\id}{\textrm{id}}


%%
%% Preamble
%%
% \documentclass{<something>} must begin each LaTeX document
\documentclass[12pt,twoside]{reedthesis}
% Packages are extensions to the basic LaTeX functions. Whatever you
% want to typeset, there is probably a package out there for it.
% Chemistry (chemtex), screenplays, you name it.
% Check out CTAN to see: http://www.ctan.org/
%%
\usepackage{graphicx,latexsym} 
\usepackage{amssymb,amsthm,amsmath}
\usepackage{longtable,booktabs,setspace} 
\usepackage[hyphens]{url}
\usepackage{rotating}
\usepackage{natbib}

%% Jay Kruer's packages
\usepackage{hyperref}
\usepackage{bbm, stmaryrd}
\hypersetup{
  colorlinks,
  allcolors=black,
  hidelinks,
}
\usepackage{luatexja} % for memes



%%% Jay Kruer's misc preamble
%%% TODO: move this to its own file
\newtheorem{definition}{Definition}
\newtheorem{example}{Example}
\newtheorem{remark}{Remark}
\newtheorem{recall}{Recall}
\newtheorem{theorem}{Theorem}
\newtheorem{joke}{Joke}

% Comment out the natbib line above and uncomment the following two lines to use the new 
% biblatex-chicago style, for Chicago A. Also make some changes at the end where the 
% bibliography is included. 
% \usepackage{biblatex-chicago}
% \bibliography{thesis}

% \usepackage{times} % other fonts are available like times, bookman, charter, palatino

\title{慣性ドリフト: From 0 to Normalization by Gluing in 4.9 seconds\\ A Brisk Drift through Categorical Semantics of Lambda Calculi}
\author{Jay Kruer}
% The month and year that you submit your FINAL draft TO THE LIBRARY (May or December)
\date{December 2021}
\division{Mathematics and Natural Sciences}
\advisor{Ang\'elica Osorno}
\altadvisor{James (Jim) Fix}

% If you have two advisors for some reason, you can use the following
% \altadvisor{Your Other Advisor}
%%% Remember to use the correct department!
\department{Mathematics}
% if you're writing a thesis in an interdisciplinary major,
% uncomment the line below and change the text as appropriate.
% check the Senior Handbook if unsure.
% \thedivisionof{The Established Interdisciplinary Committee for}
% if you want the approval page to say "Approved for the Committee",
% uncomment the next line
% \approvedforthe{Committee}

\setlength{\parskip}{0pt}
%%
%% End Preamble
%%
%% The fun begins:
\begin{document}

\maketitle
\frontmatter % this stuff will be roman-numbered
\pagestyle{empty} % this removes page numbers from the frontmatter

% Acknowledgements (Acceptable American spelling) are optional
% So are Acknowledgments (proper English spelling)
\chapter*{Acknowledgements}
Ms. Kim, Hyong Rhew, Mrs. Leitsch, Mr. Raveli (was that his name? gotta find out
somehow), Jim Fix, Ang\'elica Osorno, Irena Swanson, [TODO]

% The preface is optional
% To remove it, comment it out or delete it.
\chapter*{Preface}
This is an example of a thesis setup to use the reed thesis document class.



\chapter*{List of Abbreviations}
You can always change the way your abbreviations are formatted. Play around with it yourself, use tables, or come to CUS if you'd like to change the way it looks. You can also completely remove this chapter if you have no need for a list of abbreviations. Here is an example of what this could look like:

\begin{table}[h]
  \centering % You could remove this to move table to the left
  \begin{tabular}{ll}
    \textbf{ABC}  	&  American Broadcasting Company \\
    \textbf{CBS}  	&  Columbia Broadcasting System\\
    \textbf{CDC}  	&  Center for Disease Control \\
    \textbf{CIA}  	&  Central Intelligence Agency\\
    \textbf{CLBR} 	&  Center for Life Beyond Reed\\
    \textbf{CUS}  	&  Computer User Services\\
    \textbf{FBI}  	&  Federal Bureau of Investigation\\
    \textbf{NBC}  	&  National Broadcasting Corporation\\
  \end{tabular}
\end{table}


\tableofcontents
% if you want a list of tables, optional
\listoftables
% if you want a list of figures, also optional
\listoffigures

% The abstract is not required if you're writing a creative thesis (but aren't they all?)
% If your abstract is longer than a page, there may be a formatting issue.
\chapter*{Abstract}
The preface pretty much says it all.

\chapter*{Dedication}
You can have a dedication here if you wish.

\mainmatter % here the regular arabic numbering starts
\pagestyle{fancyplain} % turns page numbering back on

% The \introduction command is provided as a convenience.
% if you want special chapter formatting, you'll probably want to avoid using it altogether

\chapter*{Introduction}
\addcontentsline{toc}{chapter}{Introduction}
\chaptermark{Introduction}
\markboth{Introduction}{Introduction}
% The three lines above are to make sure that the headers are right, that the intro gets included in the table of contents, and that it doesn't get numbered 1 so that chapter one is 1.

% Double spacing: if you want to double space, or one and a half 
% space, uncomment one of the following lines. You can go back to 
% single spacing with the \singlespacing command.
% \onehalfspacing
% \doublespacing

Welcome to the \LaTeX\ thesis template. If you've never used \TeX\ or \LaTeX\ before, you'll have an initial learning period to go through, but the results of a nicely formatted thesis are worth it for more than the aesthetic benefit: markup like \LaTeX\ is more consistent than the output of a word processor, much less prone to corruption or crashing and the resulting file is smaller than a Word file. While you may have never had problems using Word in the past, your thesis is going to be about twice as large and complex as anything you've written before, taxing Word's capabilities. If you're still on the fence about  using \LaTeX, read the Introduction to LaTeX on the CUS site as well as skim the following template and give it a few weeks. Pretty soon all the markup gibberish will become second nature.

\chapter{Basic Type Theory}
\chapter{Enough Categories to Get Working}
\chapter{Sus semantics of type theory: sketches and their models; algebraic theories and their algebras}
In this chapter we develop the semantics of type theory by categorical methods
using \emph{sketches}. The reader familiar with and/or traumatized by experience
with pencils of geodesics in hyperbolic geometry need not be scared away by our
terminology; our development will be easy and fun. We begin with the notion of
an \emph{elementary sketch}, which is powerful enough to describe theories with
unary operations (something \emph{like} single-variable functions) and multiple
\emph{sorts} (which you may, for now, regard as a type, set, object, etc.) This
already gives us enough power to encode some familiar algebraic structures like
monoids, modules over a ring, rings themselves, among others. With their
multiple sorts, sketches also allow us to encode structures not frequently
encountered in an algebra course, such as finite automata. Later we will
generalize the notion of elementary sketch to that of \emph{algebraic theory}.
Algebraic theories are known more famously as \emph{Lawvere theories} after
categorical logic and model theory superstar (and former Reed College
professor!) William Lawvere. \emph{Algebraic theories} allow for any operations
over any number of inputs and so grow the capabilities of our development to
encoding more grotesque mathematical objects such as non-trivial type theories.
This chapter is strictly expository in nature. Much of the following
presentation draws heavily from Paul Taylor's \emph{Practical Foundations of
  Mathematics} (\cite{taylor_practical_1999}.) Our humble contribution is to
flesh out some of his examples, add some of our own, and make parts of the
presentation more palatable and quickly digestible to an audience less
sophisticated than the usual consumer of Taylor's book.

\section{Elementary sketches and their models}
\subsection{An algebraic prelude}
We begin by recalling the notion of an \emph{action} of, say, a group or a
monoid from algebra. Actions are, from our perspective, a way of giving meaning,
or \emph{semantics} to elements of a set enjoying some algebraic structure.
\begin{definition}
  Recall that a \textbf{covariant action} of a group or monoid \((M, id, \cdot)\) on a
  set \(A\) is a binary operation \((-)_* (=) : M ‌\times A \rightarrow A\) such that
  \(\text{id}_* a = a\) and \( (g \circ f)_* a = g_* (f_* a) \). We can similarly a
  define the notion of a \textbf{contravariant action} which acts in the order
  opposite of that in the composition.
\end{definition}
For example, in algebra we learn that the dihedral group of order 8, written
$D_{4}$, acts on the square (with uniquely identified points) by reflections and
rotations (TODO: cite https://groupprops.subwiki.org/wiki/Dihedral\_group:D8);
each of the operations encoded by the action results in the same image of the
square up to ignoring the unique identity of the points we started with. This
action gives geometric meaning to each of the group elements, and was even used
in the first day of the author's algebra class to explain the algebraic
mechanics of the group itself; discovering which elements of the group cancel
each other out is done by geometric experimentation using a square with uniquely
colored vertices. Similarly, the symmetric groups $S_{n}$ act on lists of length
$n$ by permutation of the list elements. In this case, the action can be even
more trivially defined. We now turn to the definition of a property of actions:
\emph{faithfulness}.

\begin{definition} A \textbf{faithful} action $(-)_{*}$ is one for which things
  are semantically equal (or, act the same) only when they are syntactically
  equal (or, are the same as far as your eyeballs are concerned.) More precisely
  rendered, we require: \[ \forall (a:A)\ldotp f_* a = g_* a \Longrightarrow f = g \]
\end{definition}

It can now be seen that the critical property enjoyed by the natural action of
$D_{8}$ which enabled our use of cute little square cutouts is faithfulness. If
the action were not faithful, determining which operations in $D_{8}$ are
inverses would not be so easy as printing out a square and plugging away,
because we may have an action which may not take only the identity element to
the leave-everything-in-place action on the square.


Having gesticulated that actions gives groups and monoids their meaning, we turn
to the development of some new notions that will allow us to define give actions
generalize both algebraic structures and their actions to new settings not
usually considered formally. % TODO: fix this phrasing
The corresponding notions are, respectively, sketches and their models.

\subsection{Elementary sketches}
As promised, we begin with the notion of an \emph{elementary sketch} which
allows us to, among other things, specify structures beyond those treated in
standard algebra.
\begin{definition}
  An \textbf{elementary sketch} is comprised of the following data:
  \begin{enumerate}
    \item a collection \(X, Y, Z, \dots \) of named \textbf{base types} or
          \textbf{sorts}
    \item a \textbf{variable} \(x:X\) for each occurrence of each named sort.
    \item a collection of \textbf{unary operation-symbols} or
          \textbf{constructors} \(\tau\) having at most one variable. As a
          clarifying example: when sketching type theories, we will write
          \( x:X \vdash \tau(x) : Y \).
    \item a collection of equations or \textbf{laws} of the
          form: \[ \tau_n (\tau_{n-1}(\cdots \tau_2 (\tau_1 (x))\cdots )) = \sigma_m (\sigma_{m-1}(\cdots \sigma_2 (\sigma_1 (x))\cdots )) \]
  \end{enumerate}
\end{definition}
We will discuss the generality provided by this definition after some
intervening examples. One of the simplest examples is the sketch of a monoid on some set $S$:
\begin{enumerate}
  \item The collection of sorts is the singleton \( \{M\} \).
  \item The collection of variables is \( \{m:M\} \).
  \item The collection of operation symbols is the set \( S \). Each has as its signature \( M \rightarrow M \)
  \item No equations are imposed.
\end{enumerate}

To really buy that this sketch generates a free monoid, we begin with an
intervening definition of a notion we will get a lot of mileage out of in this
thesis.
\begin{definition}
  Given an elementary sketch, a \textbf{term} $x : \Gamma \vdash X$ is a string of
  composable unary operation-symbols applied to a variable $\gamma : \Gamma$ as in:
  \( \tau_{n} (\tau_{n-1} (\cdots (\tau_{2}(\tau_{1}(\gamma))))) \). Composable unary-operation
  symbols are ones which have compatible domains and codomains in the usual
  sense as in set theory.
\end{definition}
We now propose a more precise version of the above claim: the terms of the
sketch defined above form the elements of a free monoid \( S \). Before we can
continue, we should decide what the composition should look like. The definition
of composition of terms is by substitution: for a term \( \sigma : \Delta \vdash \Gamma\) and some
terms \( \tau_{i}\) with \( \tau_{n} : \? \vdash \Zeta \) we define
\[ (\tau_{n} (\tau_{n-1} (\cdots (\tau_{2}(\tau_{1}(\gamma)))))) \circ \sigma = \tau_{n} (\tau_{n-1} (\cdots (\tau_{2}(\tau_{1}(\sigma(\delta)))))) : \Delta \vdash \Zeta \].
The reader will recall from our early discussions of basic type theory that
substitution is associative [TODO]; our sketch gets that monoid axiom for free. The
identity term is given by zero composable unary operation-symbols applied to a
variable. It's just a variable; when composing the identity with a term of the
appropriate type, we end up getting exactly that term [TODO: say more here, or maybe
flesh this out.]

  [transition to discussion of models]
  %   The definition of an elementary sketch allows for multiple \emph{sorts}. All
  % the common examples from algebra which we have seen so far only make use of a
  % single sort; in particular, all strings of elements of the sketches we've
  % looked at so far are composable. For an example of a structure for which this
  % is not the case, and therefore one where the notion of sketch gives us a new
  % opportunity for formalization, we shall turn to an example drawing from the
  % theory of finite automata in computability theory. [TODO: finite automata
  % example here.]

  % To really begin to buy that this sketch encodes the free monoid on the set S,
  % it helps to consider a \emph{model}, which is in some sense a way of giving a
  % sketch meaning in terms of the ambient mathematical universe.
  \begin{definition}
    A \textbf{model} (also known as an algebra, an interpretation, a covariant
    action) of an elementary sketch is comprised of:
    \begin{enumerate}
      \item an assignment of a set $A_X$ to each sort $X$ and
      \item an assignment of a function $\tau_* : A_X \rightarrow A_Y$ for each
            operation-symbol of the appropriate arity such that:
      \item each law is preserved; i.e., for each law as before we have
            \[ {\tau_n}_* ({\tau_{n-1}}_* (\cdots {\tau_2}_* ({\tau_1}_* (x))\cdots )) = {\sigma_m}_* ({\sigma_{m-1}}_*(\cdots {\sigma_2}_* ({\sigma_1}_* (x))\cdots )) \]
            that is, the covariant action on operation-symbols is faithful in
            the sense defined above.
    \end{enumerate}
  \end{definition}

  % We now will continue by giving the free monoid sketch a model:
  % \begin{example}
  %   \begin{enumerate}
  %     \item To the single sort \( M \) we assign the set \( S \)
  %     \item To the operation symbols \( s \in S \), we assign the function
  %           \( S \rightarrow S \) which takes \( t \in S\) to \( s \cdot t \); namely, it acts
  %           by left-multiplication.
  %     \item The laws are vacuosly preserved.
  %   \end{enumerate}
  % \end{example}

  \begin{definition}
    Given an elementary (unary) sketch, the \textbf{clone} at \( (\Gamma, X) \) is
    the set \( \text{Cn}_\mathcal{L} (\Gamma, X) \) of all the \textbf{terms} of sort
    $X$ assuming a single variable of sort $\Gamma$.
  \end{definition}

  It can be shown that the clones of a sketch form (the sets for) a model of a sketch. In particular, it can be shown that the sketch acts covariantly on the set of its clones:
  \begin{theorem}
    Every elementary sketch has a faithful covariant action on its clones
    \(\mathcal{H}_{X} = \cup_{\Gamma} \text{Cn}_{\mathcal{L}}(\Gamma,X)\) by
    sequencing with the operation symbol. Substitution for the (single)
    variable in a term gives a faithful contravariant action on
    \(\mathcal{H}^{Y} = \cup_{\Theta} \text{Cn}_{\mathcal{L}}(Y,\Theta)\).
  \end{theorem}
  \begin{proof}
    The actions of \(\tau : X \rightarrow Y\) on \(\text{Cn}_\mathcal{L}(\Gamma,X) \subseteq \mathcal{H}_{X}\) and \(\text{Cn}_\mathcal{L}(Y,\Theta) \subseteq \mathcal{H}^{Y}\) are given by:
    \begin{itemize}
      \item \(\tau_{*}a_{n}(\cdots a_{2}(a_{2}(\sigma))\cdots) = \tau(a_{n}(\cdots(a_{2}(a_{1}(\sigma))))) \in \text{Cn}_{\mathcal{L}}(\Gamma,Y)\)
      \item \(\tau_{*}\zeta_{m}(\cdots \zeta_{2}(\zeta_{1}(y))) = \zeta_{m}(\cdots\zeta_{2}(\zeta_{1}(\tau(x)))) \in \text{Cn}_{\mathcal{L}}(X,\Theta)\)
    \end{itemize}
    where \(\sigma : \Gamma, x : X, \text { and }, y:Y\). Covariance of the former is
    clear. Contravariance of the former follows by considering the behavior of
    substitutions in sequence. TODO: flesh out this explanation; I don't really
    remember the details any longer.
  \end{proof}

  Recalling our example sketch of a monoid, the substance of this covariant
  action morally amounts to saying that the monoid acts on itself by left
  multiplication (here it is actually juxtaposition plus some parentheses). This
  is essentially ``the same'' action that groups enjoy as guaranteed by Cayley's
  theorem.

  \begin{joke}
    A couple of PL theorists walk into a Michelin starred restaurant. The menu
    reads in big blackboard bold letters ``NO SUBSTITUTIONS''. They promptly
    leave.
  \end{joke}

\subsection{Familiar examples}
\section{Models and their categories; or, categories and their models}

\subsection{The category of contexts and substitutions}
\emph{Note:} I need to make sure when defining the category of context and substitution for a sketch by means of giving its generating morphisms that I also define the equality on morphisms. In my slides I missed this, but the correct presentation is given in Taylor.

\subsection{Models are just functors with the right codomain}

\newcommand{\clone}[3]{{\text{Cn}_{#1}(#2,#3)}}

\subsection{Example morphisms in the syntactic category}
A natural question for the operationally-minded reader to ask after having seen
the definition of the syntactic category is: how does all this ornate structure
encode terms in the calculus I'm interested in? Let us ask instead a more
precise question: how do we represent by a substitution a term \(\Gamma \vdash t : T\)?
For such a term, there is a canonical substitution (morphism of contexts)
\( \Gamma \xrightarrow[]{[t/x]} \Gamma,x:X \) which ``picks'' that term in X. Here $[t/x]$
is an explicit encoding/formula for the substitution inserting $t$ anywhere it
sees $x$. The ordering of the codomain and domain here are confusing, but the
contravariant base change functor, which lifts this encoding to a real
substitution \emph{operation} clarifies things; we have:
\( [t/x]^{*} : \clone{}{\Gamma, x:X}{T} \longrightarrow \clone{}{\Gamma}{T} \). In words, the
substitution operation takes a term of type $T$ under $\Gamma$ and an additional free
variable $x:X$ and gives us a term of type $T$ under just $\Gamma$; we reduce our
assumption set by filling in one of the assumptions with some concrete evidence,
namely the (syntactic) term $t$. In the special case of a closed (syntactic)
term $t$, we have \( [t/x]^{*} : \clone{}{x:X}{T} \longrightarrow \clone{}{\emptyset}{T}\).

\section{Algebraic theories and their algebras}
\begin{example}[Algebraic theory of \emph{ring} TODO: fix formatting]
  \begin{enumerate}
    \item Sorts: \( \mathbbm{1}, S, S, S \). The variables for each sort are
          \( \_ \) and \( x, y, z\) respectively.

    \item Operations: \[
          \cdot : S \times S \rightarrow S,
          0 : \mathbbm{1} \rightarrow S,
          1 : \mathbbm{1} \rightarrow S,
          - : S \rightarrow S\]

        \item Laws: \begin{enumerate}
          \item \( +(x,y) = +(y,x) \)
          \item \( +(0(\_), x) = x \)
          \item \( +(x, -(x)) = 0(\_) \)

          \item \( \cdot(x,y) = \cdot(y,x) \)
          \item \( \cdot(1(\_), x) = x \)

          \item \( \cdot(x, +(y,z)) = +(\cdot(x,y), \cdot(x,z))\)
    \end{enumerate}
  \end{enumerate}
\end{example}
\section{An algebraic theory: the (simply typed) lambda calculus}


\chapter{Enough Topoi to Grok Glueing}
\chapter{Grokking Glueing}
\chapter{Glueing for Normalization}

\chapter*{Conclusion}
\addcontentsline{toc}{chapter}{Conclusion}
\chaptermark{Conclusion}
\markboth{Conclusion}{Conclusion}
\setcounter{chapter}{4}
\setcounter{section}{0}

Here's a conclusion, demonstrating the use of all that manual incrementing and table of contents adding that has to happen if you use the starred form of the chapter command. The deal is, the chapter command in \LaTeX\ does a lot of things: it increments the chapter counter, it resets the section counter to zero, it puts the name of the chapter into the table of contents and the running headers, and probably some other stuff. 

So, if you remove all that stuff because you don't like it to say ``Chapter 4: Conclusion'', then you have to manually add all the things \LaTeX\ would normally do for you. Maybe someday we'll write a new chapter macro that doesn't add ``Chapter X'' to the beginning of every chapter title.

\section{More info}
And here's some other random info: the first paragraph after a chapter title or section head \emph{shouldn't be} indented, because indents are to tell the reader that you're starting a new paragraph. Since that's obvious after a chapter or section title, proper typesetting doesn't add an indent there. 


% If you feel it necessary to include an appendix, it goes here.
\appendix
\chapter{The First Appendix}
\chapter{The Second Appendix, for Fun}


% This is where endnotes are supposed to go, if you have them.
% I have no idea how endnotes work with LaTeX.

\backmatter % backmatter makes the index and bibliography appear properly in the t.o.c...

% if you're using bibtex, the next line forces every entry in the bibtex file to be included
% in your bibliography, regardless of whether or not you've cited it in the thesis.
\nocite{*}

% Rename my bibliography to be called "Works Cited" and not "References" or ``Bibliography''
% \renewcommand{\bibname}{Works Cited}

% \bibliographystyle{bsts/mla-good} % there are a variety of styles available;
% \bibliographystyle{plainnat}
% replace ``plainnat'' with the style of choice. You can refer to files in the bsts or APA 
% subfolder, e.g. 
\bibliographystyle{APA/apa-good}  % or
\bibliography{thesis}
% Comment the above two lines and uncomment the next line to use biblatex-chicago.
% \printbibliography[heading=bibintoc]

% Finally, an index would go here... but it is also optional.
\end{document}
