\documentclass[12pt,twoside]{reedthesis}

\usepackage{graphicx,latexsym}
\usepackage{amssymb,amsthm}
\usepackage{longtable,booktabs,setspace} 
\usepackage[hyphens]{url}
\usepackage{rotating}
\usepackage{hyperref}
\usepackage{outlines}

% font stuff
\usepackage{bbm, stmaryrd}
\usepackage{unicode-math}
\usepackage{luatexja} % for memes

\usepackage{amsmath}
\usepackage{mathtools}
\usepackage{braket}
\usepackage{epigraph} % funny quotes
\usepackage{quiver} % lazy diagrams

\usepackage[
backend=biber,
style=alphabetic,
citestyle=alphabetic
]{biblatex} % better citation style
\addbibresource{thesis.bib}

\hypersetup{
  colorlinks,
  allcolors=black,
  hidelinks,
}

\theoremstyle{definition}
\newtheorem{definition}{Definition}
\newtheorem{example}{Example}
\newtheorem{joke}{Joke}
\newtheorem{notation}{Notation}

\theoremstyle{remark}
\newtheorem{remark}{Remark}
\newtheorem{recall}{Recall}


\theoremstyle{plain}
\newtheorem{theorem}{Theorem}

%%% From user4686 on the TeX stackexchange. Thank you!
%%% https://tex.stackexchange.com/a/289552
\newcommand*\autoop{\left(}
\newcommand*\autocp{\right)}
\newcommand*\autoob{\left[}
\newcommand*\autocb{\right]}
\DeclareRobustCommand*\{{\ifmmode \left\lbrace \else \textbraceleft \fi }
\DeclareRobustCommand*\}{\ifmmode \right\rbrace \else \textbraceright \fi }
\AtBeginDocument {%
   \mathcode`( 32768
   \mathcode`) 32768
   \mathcode`[ 32768
   \mathcode`] 32768
   \begingroup
       \lccode`\~`(
       \lowercase{%
   \endgroup
       \let~\autoop
   }\begingroup
       \lccode`\~`)
       \lowercase{%
   \endgroup
       \let~\autocp
   }\begingroup
       \lccode`\~`[
       \lowercase{%
   \endgroup
       \let~\autoob
   }\begingroup
       \lccode`\~`]
       \lowercase{%
   \endgroup
       \let~\autocb
   }}

\delimiterfactor 1001

\makeatletter
% for amsmath "compatibility" (not sophisticated)
% \usepackage{amsmath}
\AtBeginDocument {%
          \def\resetMathstrut@{%
           \setbox\z@\hbox{\the\textfont\symoperators\char40}%
           \ht\Mathstrutbox@\ht\z@ \dp\Mathstrutbox@\dp\z@}%
}%
\makeatother

\newcommand{\term}{\emph} % TODO: make a tool to grab all the emph's from my
                          % document and optionally add them to the glossary
\newcommand{\id}{\textrm{id}}


% Comment out the natbib line above and uncomment the following two lines to use the new 
% biblatex-chicago style, for Chicago A. Also make some changes at the end where the 
% bibliography is included. 
% \usepackage{biblatex-chicago}
% \bibliography{thesis}

% \usepackage{times} % other fonts are available like times, bookman, charter, palatino

\title{慣性ドリフト: From 0 to Normalization by Gluing in 4.9 seconds\\ A Brisk Drift through Categorical Semantics of Lambda Calculi}
\author{Jay Kruer}
% The month and year that you submit your FINAL draft TO THE LIBRARY (May or December)
\date{December 2021}
\division{Mathematics and Natural Sciences}
\advisor{Ang\'elica Osorno}
\altadvisor{James (Jim) Fix}

\department{Mathematics}
% if you're writing a thesis in an interdisciplinary major,
% uncomment the line below and change the text as appropriate.
% check the Senior Handbook if unsure.
% \thedivisionof{The Established Interdisciplinary Committee for}
% if you want the approval page to say "Approved for the Committee",
% uncomment the next line
% \approvedforthe{Committee}

\setlength{\parskip}{0pt}

\begin{document}

\maketitle
\frontmatter % this stuff will be roman-numbered
\pagestyle{empty} % this removes page numbers from the frontmatter

% Acknowledgements (Acceptable American spelling) are optional
% So are Acknowledgments (proper English spelling)
\chapter*{Acknowledgements}
% My parents, Jim Fix, Ang\'elica Osorno, Irena Swanson, Kyle Ormsby, Amal Ahmed,
% Jaclyn, Nick Chaiyachakorn, Ms.\ Kim, 柳老師 (Hyong Rhew), Mrs.\ Leitsch, Mr.\
% Raveli, Noah Koster, Shulav, Aditya, Francis, Alec, Joebob, Jit, Gabe and Ciara,
% Murali Vijayaraghavan, 吳老師, Ian Desai, Becca, Andres

% The preface is optional
% To remove it, comment it out or delete it.
\chapter*{Preface}
\epigraph{And further, by these, my son, be admonished: of making many books
  there is no end; and much study is a weariness of the flesh.}{Ecclesiastes
  12:12, KJV.}

This is an example of a thesis setup to use the reed thesis
document class.

\chapter*{List of Abbreviations}
You can always change the way your abbrevias are formatted. Play around with it yourself, use tables, or come to CUS if you'd like to change the way it looks. You can also completely remove this chapter if you have no need for a list of abbreviations. Here is an example of what this could look like:

\begin{table}[h]
  \centering % You could remove this to move table to the left
  \begin{tabular}{ll}
    \textbf{TMA}  	&  Too Many Abbreviations
  \end{tabular}
\end{table}


\tableofcontents
% if you want a list of tables, optional
\listoftables
% if you want a list of figures, also optional
\listoffigures

% The abstract is not required if you're writing a creative thesis (but aren't they all?)
% If your abstract is longer than a page, there may be a formatting issue.
\chapter*{Abstract}
The preface pretty much says it all.

\chapter*{Dedication}
You can have a dedication here if you wish.

\mainmatter% here the regular arabic numbering starts
\pagestyle{fancyplain} % turns page numbering back on

% The \introduction command is provided as a convenience.
% if you want special chapter formatting, you'll probably want to avoid using it altogether

\chapter*{Introduction}
\addcontentsline{toc}{chapter}{Introduction}
\chaptermark{Introduction}
\markboth{Introduction}{Introduction}
% The three lines above are to make sure that the headers are right, that the intro gets included in the table of contents, and that it doesn't get numbered 1 so that chapter one is 1.

% Double spacing: if you want to double space, or one and a half 
% space, uncomment one of the following lines. You can go back to 
% single spacing with the \singlespacing command.
% \onehalfspacing
% \doublespacing

Welcome to the \LaTeX\ thesis template. If you've never used \TeX\ or \LaTeX\
before, you'll have an initial learning period to go through, but the results of
a nicely formatted thesis are worth it for more than the aesthetic benefit:
markup like \LaTeX\ is more consistent than the output of a word processor, much
less prone to corruption or crashing and the resulting file is smaller than a
Word file. While you may have never had problems using Word in the past, your
thesis is going to be about twice as large and complex as anything you've
written before, taxing Word's capabilities. If you're still on the fence about
using \LaTeX, read the Introduction to LaTeX on the CUS site as well as skim the
following template and give it a few weeks. Pretty soon all the markup gibberish
will become second nature.

\chapter{Basic Type Theory}
\chapter{Enough Categories to Get Working}
\chapter{Sus semantics of type theory: sketches and their models; algebraic theories and their algebras}

In this chapter we develop tools for reasoning about (syntactic)
\emph{theories}, which are in some sense ``notions of'' abstract structure.
Examples of theories include the theory of rings, and simple type theory. To
discuss how we write down our theories, we need another level of abstraction. We
will work with several \emph{notions of} (syntactic) theory, and we will more
laconically refer to a notion of syntactic theory as a \emph{doctrine}. A
\emph{doctrine} is something like a meta-framework specifying how we are to
write down a theory. The first doctrine we will consider is that of the
\emph{elementary sketch}. % The reader familiar with and/or traumatized by
% experience with pencils of geodesics in hyperbolic geometry need not be scared
% away by this terminology; there is nothing non-Euclidean afoot here.
The doctrine of the \emph{elementary sketch} allows us to write down theories
involving \emph{unary} operations (those defined over a single argument.) This
restriction on the arity of operations turns out to be quite limiting. To
address this, we will later upgrade the doctrine of elementary sketches to the
doctrine of \emph{algebraic theories} which allow encoding operations with any
finite number of arguments, thus covering a broad variety of theories. Algebraic
theories are known more famously as \emph{Lawvere theories} after categorical
logic superstar (and former Reed College professor!) William Lawvere who
originally studied them while building his functorial treatment of universal
algebra. Keeping with our sketchy terminology and emphasizing the doctrinal
upgrade, algebraic theories are also called \emph{finite product sketches}. As
an example of the strength of algebraic theories, we will show how to write down
(as an algebraic theory) what it means to be a ring, with no reference to sets
or functions. In the following chapter, we will use the doctrine of algebraic
theories to develop categorical semantics of the lambda calculus. This chapter
is strictly expository in nature. Much of the following presentation draws
heavily from Paul Taylor's \emph{Practical Foundations of Mathematics}~\cite{taylor_practical_1999}. Our humble contribution is to flesh out some of
his examples, add some of our own, and make parts of the presentation more
palatable and quickly digestible to the reader already acquainted with basic
category theory and type theory.

\section{Elementary sketches and their models}
\subsection{An algebraic prelude}
We begin by recalling from algebra the notion of an \emph{action} of, say, a
group or a monoid. Actions are, from our perspective, a way of giving meaning,
or \emph{semantics} to elements of a set which enjoys some algebraic structure.
\begin{definition}\label{def:covariant action}
  Recall that a \textbf{covariant action} of a group or monoid \((M, id, \cdot)\) on
  a set \(A\) is a binary operation \((-)_* (=) : M ‌\times A \rightarrow A\) such that
  \(\text{id}_* a = a\) and \( (g \circ f)_* a = g_* (f_* a) \). We can similarly a
  define the notion of a \textbf{contravariant action} which similarly requires
  the identity action to do nothing, but instead flips the order of action for
  compositions: $(g \circ f)^{*}a = f^{*}(g^{*} a)$
\end{definition}

For example, in algebra we learn that the dihedral group of order 8, written
$D_{4}$, acts on the square (with uniquely identified points) by reflections and
rotations\footnote{https://groupprops.subwiki.org/wiki/Dihedral\_group:D8}; each
of the operations encoded by the action results in the same image of the square
up to ignoring the unique identity of the points we started with. This action
gives geometric meaning to each of the group elements, and was used in the first
day of the author's algebra class to explain the algebraic mechanics of the
group itself; discovering which elements of the group are inverse to one another
is done by geometric experimentation using a square with uniquely colored
vertices. Similarly, the symmetric groups $S_{n}$ act on lists of length $n$ by
permutation of the list elements. In this case, the action can be even more
trivially defined. We now turn to the definition of an important property of
actions: \emph{faithfulness}.

\begin{definition}\label{def:faithful}
  A \textbf{faithful} action $(-)_{*}$ is one for which things are
  \emph{semantically} equal (or, act the same) only when they are
  \emph{syntactically} equal (or, \emph{are} the same as far as your eyeballs
  are concerned.) More precisely rendered, we
  require: \[ \forall (a:A)\ldotp f_* a = g_* a \Longrightarrow f = g \]
\end{definition}

It can now be seen that the crucial property enjoyed by the natural action of
$D_{8}$ on the square which enabled our use of paper cutouts in studying the
group is faithfulness. If the action were not faithful, determining which
operations in $D_{8}$ are inverses would not be so easy as printing out a square
and plugging away, because we may (among other catastrophes) be working with an
action which may not take only the identity element to the
leave-everything-in-place operation on the square.

Having gesticulated that actions gives groups and monoids their meaning, we turn
to the development of the doctrines of \emph{elementary sketch} and
\emph{algebraic theory} which will allow us to generalize both sets with
algebraic structures and their actions to new settings.

\subsection{Elementary sketches}
As promised, we begin with the definition.
\begin{definition}\label{def:elem sketch}
  An \textbf{elementary sketch} is comprised of the following data:
  \begin{enumerate}
    \item a collection \(X, Y, Z, \dots \) of named \textbf{base types} or
          \textbf{sorts}
    \item a \textbf{variable} \(x:X\) for each occurrence of each named sort.
    \item a collection of \textbf{unary operation-symbols} or
          \textbf{constructors} \(\tau\) having at most one variable. As a
          clarifying example: when sketching type theories, we will write
          \( x:X \vdash \tau(x) : Y \).
    \item a collection of equations or \textbf{laws} of the
          form: \[ \tau_n (\tau_{n-1}(\cdots \tau_2 (\tau_1 (x))\cdots )) = \sigma_m (\sigma_{m-1}(\cdots \sigma_2 (\sigma_1 (x))\cdots )) \]
  \end{enumerate}
\end{definition}

We will discuss the generality provided by this definition after some
intervening examples. One of the simplest examples is the sketch of a (free)
monoid on some set $S$:

\begin{example}[Sketch of a (free) monoid]\label{ex:monoid sketch}
  The requisite data for the sketch are as follows:
  \begin{enumerate}
    \item The collection of sorts is the singleton \( \{M\} \).
    \item The collection of variables is \( \{m:M\} \).
    \item The collection of operation symbols is the set \( S \). Each has as its signature \( M \rightarrow M \)
    \item No equations are imposed.
  \end{enumerate}
\end{example}

To really buy that this sketch generates a free monoid, we need an intervening
definition of a concept we will get a lot of mileage out of in this thesis. The
idea should already be familiar from our study of type theory, despite the
drastically simplified setting.

\begin{definition}\label{def:term}
  Given an elementary sketch, a \textbf{term} $x : \Gamma \vdash X$ is a string of
  composable unary operation-symbols applied to a variable $\gamma : \Gamma$ as in:
  \( \tau_{n} (\tau_{n-1} (\cdots (\tau_{2}(\tau_{1}(\gamma))))) \). Composable unary-operation
  symbols are ones which have compatible domains and codomains in the usual
  sense as in set theory.
\end{definition}

We now propose a more precise version of the above claim: the terms of the
sketch defined above form the elements of a free monoid over $S$. Before we can
continue, we should decide what our term composition will be.

\begin{definition}[Composition of terms]\label{def:term composition}
  Composition of terms is by substitution for the variable: for a term
  \( \sigma : \Delta \vdash \Gamma\) and some terms \( \tau_{i}\) with \( \tau_{n} : \cdots \vdash \Xi \) we define
  \[ (\tau_{n} (\tau_{n-1} (\cdots (\tau_{2}(\tau_{1}(\gamma)))))) \circ \sigma = \tau_{n} (\tau_{n-1} (\cdots (\tau_{2}(\tau_{1}(\sigma(\delta)))))) : \Delta \vdash \Xi. \]
\end{definition}

With our notion of composition in hand, we can now handwave an argument for our
revised claim that the terms of the sketch form the elements of a free monoid.
The reader will recall from our early discussions of basic type theory that
substitution is associative [TODO in Ch 1]. As a consequence, any elementary
sketch, including this one, satisfies that axiom for free. The identity term is
given by zero composable unary operation-symbols applied to a variable. It's
just a variable; when composing the identity term with any other term, we end up
getting exactly that original term.

This loose argument is somewhat satisfying, but we can do better. To get there,
we will first develop a notion generalizing \emph{actions} from algebra. After
doing so, we will give more concrete meaning to this sketch and complete our
intuitive handle on it.

\begin{definition}\label{def:model}
  A \textbf{model} (also known as an algebra, an interpretation, a covariant
  action) of an elementary sketch is comprised of:
  \begin{enumerate}
    \item an assignment of a set $A_X$ to each sort $X$ and
    \item an assignment of a function $\tau_* : A_X \rightarrow A_Y$ for each
    operation-symbol of the appropriate arity such that:
    \item each law is preserved; i.e., for each law as before we have
    \[ {\tau_n}_* ({\tau_{n-1}}_* (\cdots {\tau_2}_* ({\tau_1}_* (x))\cdots )) = {\sigma_m}_* ({\sigma_{m-1}}_*(\cdots {\sigma_2}_* ({\sigma_1}_* (x))\cdots )) \]
    that is, the covariant action on operation-symbols is faithful in
    the sense defined above.
  \end{enumerate}
\end{definition}

The next definition will feature prominently in our later study of type theory,
but will prove useful in studying Example~\ref{ex:monoid sketch} by forming the
sets of a ``for-free'' model for any elementary sketch.

\begin{definition}\label{def:clone}
  Given an elementary (unary) sketch, the \textbf{clone} at \( (\Gamma, X) \) is
  the set \( \text{Cn}_{\mathcal{L}} (\Gamma, X) \) of all the \textbf{terms} of sort
  $X$ assuming a single variable of sort $\Gamma$, quotiented by the laws of the
  sketch.
\end{definition}
The fact that a sketch's clones contain \emph{equivalence classes} (with respect
to the laws) of its terms will feature prominently in our later study of ideas
central to the goals of this thesis. In particular, clones alone don't allow for
any meaningful discussion of computational behavior of terms undergoing
reduction; a term's normal form and its various reducible forms are identified
in the clone.

It can be shown that the clones of a sketch form (the sets for) a model of a
sketch. In particular, it can be shown that the sketch acts covariantly on the
set of its clones:
\begin{theorem}\label{thm:clone model}
  Every elementary sketch has a faithful covariant action on its clones
  \(\mathcal{H}_{X} = \cup_{\Gamma} \text{Cn}_{\mathcal{L}}(\Gamma,X)\) by sequencing with
  the operation symbol. Substitution for the (single) variable in a term gives
  a faithful contravariant action on
  \(\mathcal{H}^{Y} = \cup_{\Theta} \text{Cn}_{\mathcal{L}}(Y,\Theta)\).
\end{theorem}
\begin{proof}
  The actions of \(\tau : X \rightarrow Y\) on
  \(\text{Cn}_{\mathcal{L}} (\Gamma,X) \subseteq \mathcal{H}_{X}\) and
  \(\text{Cn}_{\mathcal{L}} (Y,\Theta) \subseteq \mathcal{H}^{Y}\) are given by:
  \begin{itemize}
    \item \(\tau_{*}a_{n}(\cdots a_{2}(a_{2}(\sigma))\cdots) =
    \tau(a_{n}(\cdots(a_{2}(a_{1}(\sigma))))) \in
    \text{Cn}_{\mathcal{L}}(\Gamma,Y)\)
    \item \(\tau^{*}\zeta_{m}(\cdots \zeta_{2}(\zeta_{1}(y))) =
    \zeta_{m}(\cdots\zeta_{2}(\zeta_{1}(\tau(x)))) \in
    \text{Cn}_{\mathcal{L}}(X,\Theta)\)
  \end{itemize}
  where \(\sigma : \Gamma, x : X, \text { and }, y:Y\). Covariance of the
  former is clear. Contravariance of the latter follows by considering the
  behavior of substitutions in sequence.
\end{proof}

Recalling our sketch of a monoid from Example~\ref{ex:monoid sketch}, the
substance of this covariant action morally amounts to saying that the sketch
acts on its terms by left multiplication (here ``multiplication'' is actually
just juxtaposition plus some parentheses) which gives the robust version of the
handwavy argument we provided above.

\begin{joke}
  A couple of type theorists walk into a Michelin starred restaurant. The menu
  reads in blackboard bold letters $\mathbbm{``NO\, SUBSTITUTIONS''}$. They promptly leave.
\end{joke}

\subsection{The category of contexts and substitutions}
We now introduce a very special category. This category is special in both the
structure it enjoys as well as the central role it will play in the rest of the
thesis. This category goes by many names: \emph{syntactic category}, the (rather
verbose) \emph{category of contexts and substitutions}, and the elusive
\emph{classifying category}. We endeavor to explain the meaning behind each of
these names over the course of the thesis, but for now we adopt the name most
closely describing its presentation.



\begin{definition}[The category of contexts and substitutions]\label{def:syn cat}
  Given a sketch $\mathcal{L}$, the \textbf{category of contexts and substitutions}, written \( \text{Cn}^{\times}_{\mathcal{L}}\) is presented as follows:
  \begin{outline}
    \1 The objects are the contexts of \( \mathcal{L} \), i.e., finite lists of
    distinct variables and their types.

    \1 The generating morphisms are:

    \2 Single substitutions or \emph{declarations} \( [a/x] : \Gamma \rightarrow [\Gamma, x:X] \)
    for each term \( \Gamma \vdash a : X \). The direction in the signature should be
    confusing unless you're either already an expert or a total novice to type
    theory.

    \2 Single omissions or \emph{drops} \( \hat{x} : [\Gamma, x : X] \rightarrow \Gamma \) for each
    variable $x:X$.

    \1 The laws are given by an extended version of the familiar substitution
    lemma from type theory. The following laws are added for each collection of
    terms $a,b$ and distinct variables $x$ and $y$ such that $x$ does not appear
    free in $a$ and $y$ appears free in neither $a$ or $b$:
    \begin{align*}
      % declaration follow by drop does nothing
      [a/x] ; \hat{x} &= \id \\
      % successive declarations commute up to accounting for the first
      % declaration in the body of the second
      [a/x] ; [b/y]   &= [ [ a/x ]^{*} b/y ] ; [a/x] \\
      % non-overlapping declarations and drops commute
      [a/x]; \hat{y} &= \hat{y}; [a/x] \\
      % non-overlapping drops commute
      \hat{x}; \hat{y} &= \hat{y}; \hat{x} \\
      [x/y]; \hat{x}; [y/x]; \hat{y} &= \id
    \end{align*}
  \end{outline}
  We will briefly speak to the meaning of the laws. The first law says that
  binding a variable to some term and then forgetting the variable is just the
  same as doing nothing. The second law says that successive variable
  declarations commute \emph{up to accounting for the first declaration in the
    body of the second}. The third law says that \emph{non-overlapping}
  declarations and drops commute. The fourth law says that pairs of
  non-overlapping drops commute. The last law is tricky and is easier to explain
  by passing to the substitution point-of-view. Since the substitution functor
  is contravariant, this requires considering the compositions in reverse order
  as:
  \[ \hat{y}^{*}; [y/x]^{*}; \hat{x}^{*}; [x/y]^{*} = {\id}^{*} \] Rendered
  thus, this law means that introducing a free variable $y$ to the
  context\footnote{possibly having no effect if $y$ is already present},
  followed by replacing every free occurrence of $x$ with $y$, followed by
  re-introducing $x$ as a variable in the context, and then finally replacing
  every free occurrence of $y$ with $x$ is the same as doing nothing. More
  concisely at the expense of precision, renaming a free variable in a term and
  then un-renaming it results in the same term.
\end{definition}

This category, as with most in category theory, serves to allow us to define a
special class of functor. In our case, that class of functor captures what it
means to produce a model of an elementary sketch.

\subsection{Models are essentially functors with the right codomain}
\newcommand{\clone}[3]{{\text{Cn}_{#1} (#2,#3)}}
\newcommand{\cn}{\mathrm{Cn}}
\begin{theorem}[The classifying category]\label{thm:classify_elem_sketch}
  Let $\mathcal{L}$ be an elementary sketch and \( \cn_{\mathcal{L}} \) the
  category it presents. Then the models of $\mathcal{L}$ correspond to functors
  $\cn_{\mathcal{L}} \rightarrow \mathfrak{Set}$.
\end{theorem}
\begin{proof}
  Omitted. We give the proof of a more general result in
  Theorem~\ref{thm:classifying alg theory}.
\end{proof}

% \subsection{Example morphisms in the syntactic category}
% A natural question for the operationally-minded reader to ask after having seen
% the definition of the syntactic category is: how does all this ornate structure
% encode terms in the calculus I'm interested in? Let us ask instead a more
% precise question: how do we represent by a substitution a term \(\Gamma \vdash t : T\)?
% For such a term, there is a canonical substitution (morphism of contexts)
% \( \Gamma \xrightarrow[]{[t/x]} \Gamma,x:X \) which ``picks'' that term in $T$. Here
% $[t/x]$ is an explicit encoding/formula for the substitution inserting $t$
% anywhere it sees $x$. The ordering of the codomain and domain here should be
% confusing, but the contravariant base change functor, which lifts this encoding
% to a real substitution \emph{operation} clarifies things; we have:
% \( [t/x]^{*} : \clone{}{\Gamma, x:X}{T} \longrightarrow \clone{}{\Gamma}{T} \). In words, the
% substitution operation takes a term of type $T$ under $\Gamma$ and an additional free
% variable $x:X$ and gives us a term of type $T$ under just $\Gamma$; we reduce our
% assumption set by filling in one of the assumptions with some concrete evidence,
% namely the (syntactic) term $t$. In the special case of a closed (syntactic)
% term $t$, we have \( [t/x]^{*} : \clone{}{x:X}{T} \longrightarrow \clone{}{\emptyset}{T}\).

\section{Algebraic theories and their algebras}
Having defined elementary sketches, which give us a way to define multi-sorted
theories, it's obvious to request the ability to define multi-input
operations\footnote{Here's a little known statistic: At least one in two readers
  of this draft will observe that the doctrine of algebraic theory can be
  rephrased in terms of operards: algebraic theories are operads for which the
  tensor product used in forming the operation domains happens to be the plain
  ol' Cartesian product~\cite{TODO: Nlab}}. Algebraic theories generalize the
doctrine of elementary sketches and allow us to do so. As we upgrade our
doctrine to allow products, many of the notions (\emph{terms, clones, syntactic
  category, etc.}) which we developed in the simplified world of elementary
sketches will come along for the ride.

\begin{definition}[Algebraic theory]\label{def:alg theory}
  A (finitary many-sorted) \textbf{algebraic theory} $\mathcal{L}$ has
  \begin{enumerate}
    \item a collection $\Sigma$ of base types or \textbf{sorts} $X$
    \item an inexhaustible collection of variables $x_{i}:X$ of each sort;
    \item a collection of \textbf{operation symbols},
          $X_{1},\dots , X_{k} \vdash r : Y$ each having an \textbf{arity}, namely a
          list of input sorts $X_{i}$, and an output sort $Y$; and
    \item a collection of \textbf{laws}, posed as equalities between different
          terms (in the sense defined before)
  \end{enumerate}
\end{definition}

The next major concept we will introduce generalizes to algebraic theories the
notion of \emph{action} or \emph{model} we saw previously for elementary
sketches. As expected, the definition will be essentially the same up to taking
some products. Before doing so, we will give an intervening example of an
algebraic theory.
\begin{example}[Algebraic theory of \emph{ring}]\label{ex:theory of ring}

  We sketch an algebraic theory encoding the familiar structure of a ring from
  abstract algebra. The presentation should look familiar (when squinting) to
  anyone with a background in abstract algebra, except that we force the
  existence of multiplicative and additive identities by requiring any model (to
  be defined!) of this theory to provide \emph{global elements}, namely
  operations out of a distinguished sort $\mathbbm{1}$.
  \begin{enumerate}
    \item Sorts: The sorts are \( \mathbbm{1}, S\). The variable collections for
          each sort are \( \set{\square} \cup \set{\square_{i}}_{i} \) and
          \( \set{s_{i}}_{i} \cup {x,y,z} \) respectively.

    \item Operations: \begin{align*}
                        \cdot &: S \times S \rightarrow S, \\
                        + &: S \times S \rightarrow S,\\
                        0 &: \mathbbm{1} \rightarrow S,\\
                        1 &: \mathbbm{1} \rightarrow S,\\
                        - &: S \rightarrow S
                      \end{align*}

    \item Laws: \begin{align*}
                  +(x,y) &= +(y,x)\\
                  +(0(\square), x) &= x\\
                  +(x, -(x)) &= 0(\square)\\
                  \cdot(x,y) &= \cdot(y,x)\\
                  \cdot(1(\square), x) &= x\\
                  \cdot(x, +(y,z)) &= +(\cdot(x,y), \cdot(x,z))
                \end{align*}
  \end{enumerate}
\end{example}

Having given the obligatory concrete example, we now have permission to proceed
with another abstract definition: that of an \emph{$\mathcal{L}$-algebra} for an
algebraic theory:

\begin{definition}[$\mathcal{L}$-algebra]\label{def:algebra}
  Given an algebraic theory $\mathcal{L}$ and a category $C$ with finite
  products (in the sense of the universal property as treated in the chapter on
  basic category theory) an \emph{$\mathcal{L}$-algebra in $C$} is comprised of
  \begin{enumerate}
    \item an object $A_{X}$ of $C$ for each sort $X$ of $\mathcal{L}$, and
    \item for each operation symbol $X_{1}, \dots , X_{k} \vdash r : Y$, an
          assignment of a map \(r_{A} : A_{X_{1}} \times \cdots \times A_{X_{k}} \rightarrow A_{Y}\) in
          $C$.
  \end{enumerate}
  such that the assignments respect the laws of $\mathcal{L}$.
\end{definition}

We are now in good shape to give an example of an algebra (in the category of
sets) for the theory of a ring given in Example~\ref{ex:theory of ring}.
\begin{example}\label{ex:integer ring}
  % TODO: maybe need to explicate the product structure here, though I think we
  % really should get that for free with an algebraic theory.
  \begin{enumerate}
    \item For the sorts, we set $A_{S} = \mathbb{Z}$ and $A_{\mathbbm{1}} = \{\star\}$
    \item For the operations, we set
    \begin{enumerate}
      \item $\cdot_A = *$ where $*$ is the ordinary multiplication of integers
      \item $+_{A} = +$ where the second plus is ordinary addition of integers
      \item $0_{A}$ to the constant function $x \mapsto 0 \in \mathbb{Z}$
      \item $1_{A}$ to the constant function $x \mapsto 1 \in \mathbb{Z}$
      \item $-_{A}$ to the function $x \mapsto -x$ taking an integer to its additive inverse
    \end{enumerate}
    \item We wouldn't dare bore the reader by verifying all the laws, so we
          demonstrate just one. We show that the $0$ selected by the model
          indeed serves as the left identity of addition in the model.
          \begin{proof}
            \begin{align*}
              +_{A} \circ \braket{0_{A}, \id} &= (x : \{ \star \}, y : \mathbb{Z}) \mapsto 0_{A}(x) + \id(y)\\
                                          &= (x : \{ \star \}, y : \mathbb{Z}) \mapsto 0_{\mathbb{Z}} + y \\
                                          &= (x : \{ \star \}, y : \mathbb{Z}) \mapsto y\\
                                          &= \id_{\mathbbm{1}_{Z} \times S_{A}} \\
            \end{align*}

            Our proof is almost done, but we must justify that the final
            identity morphism is actually the interpretation of the variable
            (regarded as a term) $x$. This fact will be validated by results
            later in this section. In particular, Definition~\ref{def:term
              model} will give a proper treatment to the interpretation of terms
            in an algebraic theory and allow us to justify the equivalence of
            the terms $x$ and $\hat{\square}^{*} x$ by\footnote{recall from
              Definition~\ref{def:syn cat} that $\square$ is the variable we settled
              on for the sort $\mathbbm{1}$ and substitution by the hat is
              context weakening or adding the variable} way of our construction
            of the syntactic category. With the clearing of that remaining
            goalpost deferred, we have shown what is required for this law.
          \end{proof}
  \end{enumerate}
\end{example}

The reader familiar with algebra will observe that this example (at least when
fully worked out by a less lazy typist) amounts to verifying that the integers
form a ring under the standard multiplication and addition operations we learn
in elementary school. A natural next question to ask is how we might encode a
ring homomorphism in this framework. To answer this question, we (of course)
define a more general notion:

\begin{definition}[$\mathcal{L}$-algebra homomorphism]\label{def:homomorphism}
  A \emph{homomorphism} \( A \rightarrow B\) of $\mathcal{L}$-algebras $A$ and $B$ in some
  category $\mathfrak{C}$ with finite-products is an assignment to each sort $X$
  of a $\mathfrak{C}$-morphism \( \phi_{X} : A_{X} \rightarrow B_{X}\) between the
  corresponding objects in each algebra such that diagrams of the following form
  commute:
% https://q.uiver.app/?q=WzAsNCxbMCwwLCJBX3tYXzF9XFx0aW1lc1xcY2RvdHMgXFx0aW1lcyBBX3tYX2t9Il0sWzMsMCwiQV97WF8wfSJdLFswLDIsIkJfe1hfMX1cXHRpbWVzXFxjZG90c1xcdGltZXMgQl97WF9rfSJdLFszLDIsIkJfe1hfMH0iXSxbMCwxLCJyX0EiXSxbMCwyLCJcXHBoaV97WF8wfSIsMl0sWzIsM10sWzEsMywiXFxwaGlfe1hfMX1cXHRpbWVzIFxcY2RvdHMgXFx0aW1lcyBcXHBoaV97WF9rfSJdXQ==
  \[\begin{tikzcd}
      {A_{X_1}\times\cdots \times A_{X_k}} &&& {A_{X_0}} \\
      \\
      {B_{X_1}\times\cdots\times B_{X_k}} &&& {B_{X_0}}
      \arrow["{r_A}", from=1-1, to=1-4]
      \arrow["{\phi_{X_0}}"', from=1-1, to=3-1]
      \arrow[from=3-1, to=3-4]
      \arrow["{\phi_{X_1}\times \cdots \times \phi_{X_k}}", from=1-4, to=3-4]
    \end{tikzcd}\]

\end{definition}

\begin{remark}\label{rmk:alg_func}
  (The following is an observation due to Lawvere in a paper written in his time
  teaching at Reed College~\cite{lawvere_functorial_1963}.) The familiarity of
  this diagram is no mistake: indeed, by analogy to
  Theorem~\ref{thm:classify_elem_sketch}, we may understand algebras as
  product-preserving functors. A mapping between algebras then is a natural
  transformation, hence the naturality diagram in
  Definition~\ref{def:homomorphism}. We will later make this analogy more
  concrete in Theorem~\ref{thm:classifying alg theory}.
\end{remark}

\begin{remark}
  Predictably, the $\mathfrak{C}$-valued algebras and homomorphisms of an
  algebraic theory $\mathcal{L}$ form a category, called
  $\mathscr{Mod}_{\mathfrak{C}}(\mathcal{L})$.
  \begin{proof}
    Per Remark~\ref{rmk:alg_func}, we consider algebras and their homomorphisms
    as functors and natural transformations respectively. The identity morphisms
    are the identity natural transformations whose component morphisms are the
    identities of $\mathfrak{C}$. Composition of natural transformations is
    given by composition of their component morphisms, hence we may out-source
    the associativity condition to that guaranteed by the categorical structure
    of $\mathfrak{C}$.
  \end{proof}

\end{remark}

Most questions in type theory are concerned with the \emph{terms} of the theory
at hand. Normalization theorems talk about the accessibility (under some
reduction relation) of a certain class of terms from any arbitrary term.
Canonicity, a stronger property implying normalization, talks about the
accessibility of another more strict class of terms from arbitrary start terms.
These are but two examples of a broad spectrum of properties one might desire of
the terms of a theory. Considering the primacy of term properties in type
theory, it is rather strange that the notion of semantics we have built so far
makes no commentary on terms besides the action on the clones given in
Theorem~\ref{thm:clone model}. Our models so far have only given meaning to the
individual \emph{sorts} (types) and individual \emph{operation symbols}
(constructors) of the theory considered. In fact, this is enough: our models
extend canonically to contexts and substitutions and thus give meaning to terms.

\DeclarePairedDelimiter{\sem}{\llbracket}{\rrbracket}

\begin{definition}[Extending a model to terms]\label{def:term model}
  Let $A$ be an $\mathcal{L}$-algebra in a category $\mathfrak{C}$. This algebra
  extends canonically to an interpretation $\sem{-}$ of contexts by the
  following definition recursive in the structure of contexts:
  \begin{align}
    \label{eq:contexts interp}
    \sem{\varnothing} &= \mathbbm{1}_{C} \\
    \sem{\Gamma, x : X} &= \sem{\Gamma} \times A_{X}
  \end{align}
  The (overly) careful reader will complain that $\mathfrak{C}$ doesn't
  necessarily feature a terminal object, but it turns out that a terminal object
  is guaranteed\footnote{as the nullary finite product} by the finite product
  closure we imposed on $\mathfrak{C}$ in our definition of algebras. We are
  good to go.

  Recalling more from the definition of an algebra, we know that $A$ gives
  meaning to each operation symbol \( Y_{1},\dots , Y_{k} \vdash r : Z \) as a
  morphism \( r_{A} : A_{Y_{1}} \times \cdots \times A_{Y_{k}} \rightarrow A_{Z} \) and gives meaning to
  each constant \( c : Z \) by a morphism \( 1_{\mathfrak{C}} \rightarrow A_{Z} \). We can
  extend this uniquely to arbitrary terms in the context
  \( \Gamma \equiv \sem{x_{1} : X_{1}, \dots , x_{n} : X_{n}} \) by the following
  recursive definition:
  \begin{align}
    \label{eq:term interp}
    \sem{x_{i}} &: \sem{\Gamma} \equiv A_{X_{1}} \times \cdots \times A_{X_{n}} \xrightarrow{\pi_{i}} A_{X_{i}} \\
    \sem{c} &: \sem{\Gamma} \xrightarrow{<_{!}} \mathbbm{1}_{C} \xrightarrow{c_{A}} A_{Z} \\
    \sem{r(u_{1}, \dots , u_{k})} &: \sem{\Gamma} \xrightarrow{\braket{\sem{u_{1}}, \dots, \sem{u_{k}}}} A_{Y_{1}} \times \cdots \times A_{Y_{k}} \xrightarrow{r_{A}} A_{Z}
  \end{align}
  where the $\sem{u_{i}}$ are the interpretations of the sub-expressions of the
  expression in the final line, $\pi_{i}$ is the $i$th projection guaranteed to us
  by the universal property of products, and $<_{!}$ is the unique map into the
  terminal object. The angle bracket notion is used to express the product
  functor's action on morphisms in $\mathfrak{C}$. For clarity, we write out
  explicitly the composites for the reader:
  \begin{align*}
    \sem{x_{i}} &\equiv \pi_{i} \\
    \sem{c} &\equiv c_{A } \circ <_{!} \\
    \sem{r(u_{1}, \dots , u_{k})} &\equiv r_{A} \circ \braket{\sem{u_{1}}, \dots,\sem{u_{k}}}
  \end{align*}
\end{definition}

\begin{theorem}[The classifying category of an algebraic
  theory]\label{thm:classifying alg theory} Let \( \mathcal{L} \) be an
  algebraic theory. Then \begin{enumerate}
    \item $\cn_{\mathcal{L}}^{\times}$ has finite products and an $\mathcal{L}$-algebra.
    \item Let $\mathfrak{C}$ be another category with a choice of finite
    products and an $\mathcal{L}$-algebra. Then the functor $\sem{-} :
    \cn_{\mathcal{L}}^{\times} \rightarrow \mathfrak{C}$ preserves finite
    products and the $\mathcal{L}$-algebra, and is the unique such functor.
    \item Any functor \( \cn_{\mathcal{L}}^{\times} \rightarrow C \) which
    preserves finite products also preserves the $mathcal{L}$-algebra.
  \end{enumerate}
\end{theorem}
\newcommand{\ob}{\mathrm{ob}\,}
\begin{proof}\,\\
  \begin{enumerate}
    \item We first show that the syntactic category has finite products. Recall
    that the objects of the syntactic category are variable contexts \([x : X, y
    : Y, z : Z, \dots]\). For any other context \( [t : T, u : U, v : V, \dots]
    \) we have the product \[ [x : X, y : Y, z : Z, \dots] \times [t : T, u : U,
    v : V] = [x : X, y : Y, z : Z, \dots, t : T, u : U, v : V, \dots]\] That
    is, products are given by concatenation of contexts. Now the model is given
    as follows:
    \begin{enumerate}
      \item The sorts $X$ of $\mathcal{L}$ are interpreted as single variable
      contexts $[x:X] \in \ob C$ where the variable $x$ is arbitrary.
      \item The operation symbols \( X_1, X_2, \dots \vdash r : Y \) of
      $\mathcal{L}$ are interpreted as substitutions \( [r(x_1, x_2, \dots)/y] :
      [x_1:X_1, x_2:X_2,\dots] \rightarrow [y : Y]\) 
    \end{enumerate}
    \item The functor promised is precisely the one defined by
          Definition~\ref{def:term model}. TODO: we still need to verify the
          uniqueness of this functor, which Taylor himself never demonstrates.
          \item This result demands a full proof, which isn't given in Taylor.
          \begin{proof}
            Suppose we have a functor $F : \cn_{\mathcal{L}}^{\times} \rightarrow \mathfrak{C}$
            which preserves products. We will show that it preserves the
            $\mathcal{L}$-model, in particular, that it takes the interpretation
            in $\cn_{\mathcal{L}}^{\times}$ of any context $\Gamma$ to the interpretation
            of $\Gamma$ in $\mathfrak{C}$. TODO.

            \textbf{Note}: Taylor claims the proof of this is given
            in~\cite{lawvere_functorial_1963}, but I am (so far, I haven't tried
            too long) unable to identify the result in that paper, probably
            because Lawvere's formulation of algebraic theories is very
            different from Taylor's. I plan to tie up this loose end sometime
            before the end of August.
          \end{proof}
  \end{enumerate}
\end{proof}


\section{Sketching the simply typed lambda calculus}


\chapter{Enough Topoi to Grok Glueing}
\chapter{Grokking Glueing}
\chapter{Glueing for Normalization}

\chapter*{Conclusion}
\addcontentsline{toc}{chapter}{Conclusion}
\chaptermark{Conclusion}
\markboth{Conclusion}{Conclusion}
\setcounter{chapter}{4}
\setcounter{section}{0}

That's it for now.

% If you feel it necessary to include an appendix, it goes here.
\appendix
\chapter{The First Appendix}
\chapter{The Second Appendix, for Fun}


% This is where endnotes are supposed to go, if you have them.
% I have no idea how endnotes work with LaTeX.

\backmatter% backmatter makes the index and bibliography appear properly in the t.o.c...

% if you're using bibtex, the next line forces every entry in the bibtex file to be included
% in your bibliography, regardless of whether or not you've cited it in the thesis.
\nocite{*}

% Rename my bibliography to be called "Works Cited" and not "References" or ``Bibliography''
% \renewcommand{\bibname}{Works Cited}

% \bibliographystyle{bsts/mla-good} % there are a variety of styles available;
% \bibliographystyle{plainnat}
% replace ``plainnat'' with the style of choice. You can refer to files in the bsts or APA 
% subfolder, e.g.
\printbibliography[heading=bibintoc]
% \bibliographystyle{APA/apa-good}  % or
% \bibliography{thesis}
% Comment the above two lines and uncomment the next line to use biblatex-chicago.
% \printbibliography[heading=bibintoc]

% Finally, an index would go here... but it is also optional.
\end{document}
